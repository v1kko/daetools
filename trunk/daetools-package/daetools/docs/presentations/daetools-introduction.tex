%-=-=-=-=-=-=-=-=-=-=-=-=-=-=-=-=-=-=-=-=-=-=-=-=
%        DOCUMENT
%-=-=-=-=-=-=-=-=-=-=-=-=-=-=-=-=-=-=-=-=-=-=-=-=
\documentclass[compress,newPxFont,sthlmFooter]{beamer}
\usetheme{sthlm} %Luebeck
%\usecolortheme{sthlmv42}

%-=-=-=-=-=-=-=-=-=-=-=-=-=-=-=-=-=-=-=-=-=-=-=-=
%        PACKAGES
%-=-=-=-=-=-=-=-=-=-=-=-=-=-=-=-=-=-=-=-=-=-=-=-=
% \usepackage[backend=biber, style=numeric]{biblatex}
\usepackage{biblatex}
% \bibliographystyle{apalike}
\newcommand{\skipcite}[1]{} 

\usepackage[absolute,overlay]{textpos}
\usepackage{hyperref}
\usepackage[utf8]{inputenc}
\usepackage[english,serbian]{babel}
\usepackage{color}
\usepackage{chronology}
\renewcommand{\event}[3][e]{%
  \pgfmathsetlength\xstop{(#2-\theyearstart)*\unit}%
  \ifx #1e%
    \draw[fill=black,draw=none,opacity=0.5]%
      (\xstop, 0) circle (.2\unit)%
      node[opacity=1,rotate=45,right=.2\unit] {#3};%
  \else%
    \pgfmathsetlength\xstart{(#1-\theyearstart)*\unit}%
    \draw[fill=black,draw=none,opacity=0.5,rounded corners=.1\unit]%
      (\xstart,-.1\unit) rectangle%
      node[opacity=1,rotate=45,right=.2\unit] {#3} (\xstop,.1\unit);%
  \fi}%
  
\usepackage{amsmath}
\usepackage{amssymb,amsfonts,textcomp}
\usepackage{color}
\usepackage{placeins}
\usepackage{caption}
\usepackage{listings}
\usepackage{xcolor}
% \usepackage{droidmono}
\usepackage{subscript}

% \usecolortheme{default}
% \usefonttheme{professionalfonts}
% \usefonttheme[onlymath]{serif}

\definecolor{light_red}{RGB}{255, 100, 100}
\definecolor{light_green}{RGB}{100, 255, 100}

%-=-=-=-=-=-=-=-=-=-=-=-=-=-=-=-=-=-=-=-=-=-=-=-=
%        BEAMER OPTIONS
%-=-=-=-=-=-=-=-=-=-=-=-=-=-=-=-=-=-=-=-=-=-=-=-=
% \addbibresource{daetools.bib}

\setbeamertemplate{caption}{\insertcaption} 
\setbeamertemplate{caption label separator}{}

%\setbeameroption{show notes}
%\setbeamertemplate{note page}[plain]
%\setbeamertemplate{blocks}[rounded][shadow=true]
\definecolor{amethyst}{rgb}{0.6, 0.4, 0.8}
\definecolor{aogreen}{rgb}{0.0, 0.5, 0.0}
\definecolor{burgundy}{rgb}{0.5, 0.0, 0.13}
\definecolor{battleshipgrey}{rgb}{0.52, 0.52, 0.51}

\lstset{
    language=Python,
    basicstyle=\fontfamily{fdm}\scriptsize,
    keywordstyle=\color{blue},
    commentstyle=\color{aogreen},
    numberstyle=\color{battleshipgrey},
    stringstyle=\color{burgundy},
    identifierstyle=\color{black},
    frame=single,
    frameround=tttt,
    %numbers=left,
    keepspaces=true,
    showspaces=false,
    showtabs=false,
    showstringspaces=false,
    morekeywords={import, from, class, def, for, while, if, is, in, elif, else, not, and, or, print, break, 
                  continue, return, True, False, except, finally, import, lambda, pass, raise, try, None, __init__}
}

\lstset{
    language=C++,
    basicstyle=\fontfamily{fdm}\scriptsize,
    keywordstyle=\color{blue},
    commentstyle=\color{aogreen},
    numberstyle=\color{battleshipgrey},
    stringstyle=\color{burgundy},
    identifierstyle=\color{black},
    frame=single,
    frameround=tttt,
    %numbers=left,
    keepspaces=true,
    showspaces=false,
    showtabs=false,
    showstringspaces=false
}

\lstdefinestyle{gPROMS}{
    language=Python,
    basicstyle=\fontfamily{fdm}\scriptsize,
    keywordstyle=\color{blue},
    commentstyle=\color{aogreen},
    numberstyle=\color{battleshipgrey},
    stringstyle=\color{burgundy},
    identifierstyle=\color{black},
    frame=single,
    frameround=tttt,
    %numbers=left,
    keepspaces=true,
    showspaces=false,
    showtabs=false,
    showstringspaces=false,
    morekeywords={AS, EQUATION, PARAMETER, VARIABLE}
}

\lstdefinestyle{Modelica}{
    language=C++,
    basicstyle=\fontfamily{fdm}\scriptsize,
    keywordstyle=\color{blue},
    commentstyle=\color{aogreen},
    numberstyle=\color{battleshipgrey},
    stringstyle=\color{burgundy},
    identifierstyle=\color{black},
    frame=single,
    frameround=tttt,
    %numbers=left,
    keepspaces=true,
    showspaces=false,
    showtabs=false,
    showstringspaces=false,
    morekeywords={model, end, import, parameter, Real, equation, der}
}

% \setbeamertemplate{bibliography entry title}{}
% \setbeamertemplate{bibliography entry location}{}
% \setbeamertemplate{bibliography entry note}{}

%-=-=-=-=-=-=-=-=-=-=-=-=-=-=-=-=-=-=-=-=-=-=-=-=
%   PRESENTATION INFORMATION
%-=-=-=-=-=-=-=-=-=-=-=-=-=-=-=-=-=-=-=-=-=-=-=-=
\title{DAE Tools Software}
\subtitle{Introduction}
\author{D.D. Nikolić}
\institute
{
  DAE Tools Project, \url{http://www.daetools.com}
}
%\logo{\includegraphics{../images/[d][a][e]_Tools_project.png}}
\date{1 April 2016} 

\hypersetup{
pdfauthor = {Dragan D. Nikolić: dnikolic@daetools.com},
pdfsubject = {},
pdfkeywords = {},
pdfmoddate= {D:\pdfdate},
pdfcreator = {}
}

% \titlegraphic{\includegraphics{../[d][a][e]_Tools_project.png}}

% \AtBeginSection[]
% {
%   \begin{frame}
%     \frametitle{Outline}
%     \tableofcontents[currentsection]
%   \end{frame}
% }
% 
\begin{document}

% \begin{frame}
% \titlepage
% \end{frame}
\maketitle

\begin{frame}{Outline}
\tableofcontents[sectionstyle=show, 
                 subsectionstyle=hide]
\end{frame} 

\section{General Information}

\begin{frame}{What is DAE Tools?} 
Process \alert{modelling}, \alert{simulation}, and \alert{optimisation} software
\begin{itemize}
  \item Areas of application:
    \begin{itemize}
      \item Initially: chemical process industry (mass, heat and momentum transfers, chemical reactions, 
                                                  separation processes, thermodynamics, electro-chemistry)
      \item Nowadays: \alert{multi-domain}
    \end{itemize}
  \item Free/Open source software (\alert{GNU GPL})
  \item \alert{Cross-platform} (GNU/Linux, MacOS, Windows)
  \item \alert{Multiple architectures} (32/64 bit x86, ARM, ...)
\end{itemize}
\end{frame}

\begin{frame}{What is DAE Tools? (cont'd)} 
  \begin{itemize}
    \item DAE Tools \alert{is not}:
        \begin{itemize}
            \item A modelling language (such as Modelica, gPROMS, ...)
            \item An integrated software suite of data structures and routines for scientific applications (such as PETSc, Sundials, ...)
        \end{itemize}
    \item DAE Tools \alert{is}:
        \begin{itemize}
            \item A \alert{hybrid} approach between modelling and general-purpose programming languages 
            \item A higher level structure – an architectural design of interdependent software components
                  providing an API for:
                \begin{itemize}
                    \item Model development/specification
                    \item Activities on developed models (simulation, optimisation, ...)
                    \item Processing of the results
                    \item Report generation
                    \item Code generation and model exchange
                \end{itemize}
        \end{itemize}
  \end{itemize}
\end{frame}

\begin{frame}{What can be done with DAE Tools?} 
\begin{itemize}
  \item Simulation
    \begin{itemize}
      \item Steady-State 
      \item Transient
    \end{itemize}
  \item Optimisation
    \begin{itemize}
      \item Non-Linear Programming (NLP) problems
      \item Mixed Integer Non-Linear Programming (NLP) problems
    \end{itemize}
  \item Parameter estimation
    \begin{itemize}
      \item Levenberg–Marquardt algorithm
    \end{itemize}
  \item Code-generation, model-exchange, co-simulation 
    \begin{itemize}
      \item Modelica, gPROMS, Matlab, Simulink
      \item Functional Mockup Interface (FMI)
      \item C99 (for embedded systems)
      \item C++ MPI (for distributed computing) 
    \end{itemize}
\end{itemize}
\end{frame}

\begin{frame}{Types of systems that can be modelled}
  \alert{Initial value problems of implicit form}, 
  (described by systems of linear, non-linear, and (partial-)differential algebraic equations).
\begin{itemize}
  \item \alert{Continuous} with some elements of \alert{event-driven} systems 
        (discontinuous equations, state transition networks and discrete events) 
  \item \alert{Steady-state} or \alert{dynamic}
  \item With \alert{lumped} or \alert{distributed} parameters 
        (finite difference, finite volume and finite element methods)
  \item Only \alert{index-1} DAE systems at the moment
\end{itemize}
\end{frame}

\section{Motivation}

\begin{frame}
\frametitle{Why modelling software?}
In general, two scenarios:
\begin{itemize}
  \item \alert{Development} of a \alert{new} product/process/...
    \begin{itemize}
        \item Reduce the time to market (TTM)
        \item Reduce the development costs (no physical prototypes)
        \item Maximise the performance, yield, productivity, purity, ...
        \item Minimise the capital and operating costs
        \item Explore the new design options in less time and no risks
    \end{itemize}

  \item \alert{Optimisation} of an \alert{existing} product/process/...
    \begin{itemize}
        \item Increase the performance, yield, productivity, purity, ...
        \item Reduce the operating costs, energy consumption, ...
        \item Debottleneck
    \end{itemize}
\end{itemize}
\end{frame}

\begin{frame}{Why \textsc{yet another} modelling software?}
Currently available options:
\begin{enumerate}
   \item \alert{Modelling languages} (domain-specific or multi-domain) 
         (Modelica \skipcite{Fritzson-and-Engelson-1998}, Ascend \skipcite{Piela-etal-1991}, gPROMS \skipcite{Barton-and-Pantelides-1994}, 
         GAMS \skipcite{Brook-etal-1988}, Dymola \skipcite{Elmqvist-1978}, APMonitor \skipcite{APMonitor-2014})
   \item \alert{General-purpose programming languages}:
      \begin{itemize}
          \item Lower level third-generation languages such as C, C++ and Fortran
                (PETSc \skipcite{petsc}, SUNDIALS \skipcite{Hindmarsh-etal-2005})
          \item Higher level fourth-generation languages such as Python (NumPy, SciPy, Assimulo \skipcite{Assimulo-2015}), Julia etc.
          \item Multi-paradigm numerical languages 
                (Matlab \skipcite{matlab}, Mathematica \skipcite{mathematica}, Maple \skipcite{maple}, 
                Scilab \skipcite{scilab}, and GNU Octave \skipcite{octave})
      \end{itemize}
 \end{enumerate}
\end{frame}

\begin{frame}{Why \textsc{yet another} modelling software? (cont'd)}
  The advantages of the \alert{Hybrid} approach over the \alert{modelling} and \alert{general-purpose} programming languages:
    \begin{enumerate}
        \item Support for the \alert{runtime model generation}
        \item Support for the \alert{runtime simulation set-up}
        \item Support for \alert{complex runtime operating procedures}
        \item \alert{Interoperability} with the \alert{third-party software} packages (i.e. NumPy/SciPy)
        \item Suitability for \alert{embedding} and use as a \alert{web application} or \alert{software as a service}
        \item \alert{Code-generation}, \alert{model exchange} and \alert{co-simulation} capabilities  
    \end{enumerate}
\end{frame}

\begin{frame}{Additional features}
    \begin{itemize}
        \item Support for the \alert{automatic differentiation} (ADOL-C)
        \item Support for the \alert{sensitivity analysis} through the auto-differentiation capabilities
        \item Support for the \alert{parallel} computation (OpenMP, GPGPU, MPI)
        \item \alert{Interoperability} with the \alert{$3^{rd}$} party numerical software (NumPy, SciPy, ...)
        \item Support for a large number of \alert{DAE}, \alert{LA} and \alert{NLP} solvers 
        \item Support for the generation of \alert{model reports} (XML + MathML, Latex)
        \item \alert{Export} of the \alert{simulation results} to various file formats (Matlab, Excel, json, xml, HDF5, Pandas)
    \end{itemize}
\end{frame}

\section{Programming Paradigms}

\begin{frame}{The hybrid approach}
\begin{itemize}
  \item 
\end{itemize}
\end{frame}

\begin{frame}{Object-oriented modelling}
\begin{itemize}
  \item Everything is an object (models, parameters, variables, equations, state transition networks, simulations, solvers, ...) 
  \item Models are classes derived from the base daeModel class (inheriting the common functionality)
  \item Hierarchical model decomposition allows creation of complex, re-usable model definitions
  \item All Object Oriented concepts supported (such as multiple inheritance, templates, polymorphism, ...) 
        that are supported by the target language (c++, Python), except:
  \begin{itemize}
      \item Derived classes always inherit all declared objects (parameters, variables, equations, ...)  
      \item All parameters, variables, equations etc. remain public
  \end{itemize}
\end{itemize}
\end{frame}

\begin{frame}{Equation-oriented (acausal) modelling}
\begin{itemize}
  \item Equations given in an implicit form (as a residual)
    \begin{center}
      $F(\dot {x}, x, y, p) = 0$
    \end{center}
  \item Input-Output causality is not fixed:
  \begin{itemize}
    \item Increased model re-use
    \item Support for different simulation scenarios (based on a single model) by specifying different degrees of freedom
  \end{itemize}
  \item For instance, equation given in the following form:
    \begin{center}
      $x_1 + x_2 + x_3 = 0$
    \end{center}
    can be used to determine either $x_1$, $x_2$ or $x_3$ depending on what combination of variables is known:
    \begin{center}
      $x_1 = -x_2 - x_3 \; or \;  
      x_2 = -x_1 - x_3 \; or \; 
      x_3 = -x_1 - x_2$
    \end{center}
\end{itemize}
\end{frame}

\begin{frame}{Separation of model definition from activities on models}
\begin{itemize}
  \item The structure of the model (parameters, variables, equations etc.) given in the model classes ($daeModel$, $daeFiniteElementModel$) 
  \item The runtime information in the simulation class ($daeSimulation$)
  \item Single model definition, but:
  \begin{itemize}
    \item One or more different simulation scenarios
    \item One or more optimization scenarios
  \end{itemize}
\end{itemize}
\end{frame}

% \begin{frame}
% \frametitle{Hybrid continuous/discrete systems}
% \begin{block}{}
% \begin{itemize}
%   \item Modelling of continuous systems with some elements of event-driven systems 
%   \begin{itemize}
%     \item Discontinuous equations
%     \item State transition networks
%     \item Discrete events
%   \end{itemize}
% \end{itemize}
% \end{block}
% \end{frame}

% \begin{frame}
% \frametitle{Code generation}
% \begin{block}{}
% \begin{itemize}
%   \item Model export from DAE Tools to other DSL/modelling/programming languages
%   \begin{itemize}
%     \item Modelica
%     \item c99
%   \end{itemize}
% \end{itemize}
% \end{block}
% \end{frame}
% 
% \begin{frame}
% \frametitle{Model Exchange}
% \begin{block}{}
% \begin{itemize}
%   \item Support for Functional Mock-up Interface for Model Exchange and Co-Simulation (FMI): https://www.fmi-standard.org 
%   \item FMI – a tool independent standard to support both model exchange and co-simulation of dynamic models using a combination of xml-files and compiled C-code
%   \item Still in experimental phase
% \end{itemize}
% \end{block}
% \end{frame}
% 
% \begin{frame}
% \frametitle{Model reports}
% \begin{block}{}
% \begin{itemize}
%   \item Automatic model documentation
%   \item XML + MathML format
%   \item XSL transformation used to to generate HTML code and visualize reports 
%   \item Two types:
%   \begin{itemize}
%     \item Model description report (contains model definition)
%     \item Runtime report with all values and equations expanded (contains definition of the simulation)
%   \end{itemize}
% \end{itemize}
% \end{block}
% \end{frame}
% 
% \begin{frame}
% \frametitle{Model reports (cont'd)}
% \includegraphics<1>[height=0.7\paperheight]{../_static/model_report.png}
% \includegraphics<2>[height=0.7\paperheight]{../_static/runtime_model_report.png}
% \end{frame}
% 
% \begin{frame}
% \frametitle{Multi-domain}
% \begin{block}{}
% \begin{itemize}
%   \item From chemical processing industry to biological neural networks
%   \item DAE Tools is not a DSL but defines the basic modelling concepts such as models, parameters, variables, various types of equations 
%         (ordinary, differential, partial differential, discontinuous), state transition networks etc. that can be used as building blocks 
%         for a specific domain
%   \item Example: a reference implementation simulator for NineML (xml-based modelling language for describing networks of spiking neurons)
%   \item The key concepts from NineML are based on DAE Tools concepts: Neurone, Synapse, Population of neurones, Layers, Projections etc
% \end{itemize}
% \end{block}
% \end{frame}
% 
% \begin{frame}
% \frametitle{DAE Plotter}
% \begin{columns}
%   \begin{column}{0.5\paperwidth}
%     \begin{center}
%       \begin{itemize}
% 	\item 2D plots (Matplotlib)
% 	\item Animated 2D plots
% 	\item 3D plots (Mayavi)
%       \end{itemize}
%     \end{center}
%     \begin{center}
%       \includegraphics[width=0.35\paperwidth]{../_static/daeplotter.png}
%     \end{center}
%   \end{column}
%   \begin{column}{0.5\paperwidth}
%     \begin{center}
%       \includegraphics[width=0.35\paperwidth]{../_static/sample_2d_plot.png}
%     \end{center}
%     \begin{center}
%       \includegraphics[width=0.3\paperwidth]{../_static/sample_3d_plot.png}
%     \end{center}
%   \end{column}
% \end{columns}
% \end{frame}
% 
% \begin{frame}
% \frametitle{Solvers}
% \begin{block}{}
% Supported DAE solvers@
% \begin{itemize}
%   \item Sundials IDAS ({\small https://computation.llnl.gov/casc/sundials/main.html})
% \end{itemize}
% \end{block}
% 
% \begin{block}{}
% Supported FE libraries:
% \begin{itemize}
%   \item deal.II ({\small http://dealii.org})
% \end{itemize}
% \end{block}
% 
% \begin{block}{}
% Supported optimization solvers:
% \begin{itemize}
%   \item IPOPT ({\small https://projects.coin-or.org/Ipopt}) 
%   \item Bonmin ({\small https://projects.coin-or.org/Bonmin}) 
%   \item NLOPT ({\small http://ab-initio.mit.edu/wiki/index.php/NLopt}) 
% \end{itemize}
% \end{block}
% \end{frame}
% 
% \begin{frame}
% \frametitle{Solvers}
% \begin{block}{}
% Supported LA solvers:
% \begin{itemize}
%   \item Sundials dense LU, Lapack
%   \item Trilinos Amesos ({\small http://trilinos.sandia.gov/packages/amesos}) 
%   \item Trilinos AztecOO ({\small http://trilinos.sandia.gov/packages/aztecoo}) 
%   \item SuperLU SuperLU-MT ({\small http://crd.lbl.gov/~xiaoye/SuperLU/index.html}) 
%   \item Umfpack ({\small http://www.cise.ufl.edu/research/sparse/umfpack}) 
%   \item MUMPS ({\small http://graal.ens-lyon.fr/MUMPS})
%   \item CUSP ({\small http://code.google.com/p/cusp-library}) 
%   \item Intel Pardiso ({\small http://software.intel.com/en-us/articles/intel-mkl})
% \end{itemize}
% \end{block}
% \end{frame}

\section{Architecture}

\begin{frame}[plain]{The fundamental concepts/software interfaces}
\begin{columns}
    \begin{column}{0.28\paperwidth}
        \scriptsize{
        \begin{itemize}
            \item Concepts/Interfaces:
                \begin{itemize}
                    \tiny{
                        \item \textcolor{sthlmRed}{daeModel\_t}
                        \item \textcolor{sthlmRed}{daeSimulation\_t}
                        \item \textcolor{sthlmRed}{daeOptimization\_t}
                        \item \textcolor{sthlmRed}{daeBlock\_t}
                        \item \textcolor{sthlmRed}{daeDAESolver\_t}
                        \item \textcolor{sthlmRed}{daeLASolver\_t}
                        \item \textcolor{sthlmRed}{daeDataReporter\_t}
                        \item \textcolor{sthlmRed}{daeBlock\_t}
                    }
                \end{itemize}
            \item In 6 packages:
                \begin{itemize}
                \tiny{
                        \item \alert{core}
                        \item \alert{activity}
                        \item \alert{datareporting}
                        \item \alert{solvers}
                        \item \alert{logging}
                        \item \alert{units}
                    }
                \end{itemize}
        \end{itemize}
        }
     \end{column}
     
     \begin{column}{0.72\paperwidth}
         \begin{center}
             \begin{figure}
               \includegraphics[width=0.68\paperwidth]{daetools-architecture.png}      
               \caption{The architecture of \alert{DAE Tools}.}
            \end{figure}
         \end{center}
     \end{column}
   \end{columns}
\end{frame}

\begin{frame}{Package \textsc{core}}
\scriptsize
{
\begin{table}
  \caption{The key modelling concepts in the \alert{core} package.}
  \begin{tabularx}{\linewidth}{l>{\raggedright}X}
    \toprule
    \textcolor{sthlmRed}{\textbf{Concept}} & \textcolor{sthlmRed}{\textbf{Description}} \tabularnewline
    \midrule
    \textcolor{sthlmRed}{\textit{daeVariableType\_t}} & Defines a variable type that has the units, lower and upper bounds, a default value and 
                                  an absolute tolerance \tabularnewline
    \textcolor{sthlmRed}{\textit{daeDomain\_t}} & Defines ordinary arrays or spatial distributions such as structured and unstructured grids \tabularnewline 
    \textcolor{sthlmRed}{\textit{daeParameter\_t}} & Defines time invariant quantities that do not change during a simulation \tabularnewline 
    \textcolor{sthlmRed}{\textit{daeVariable\_t}} & Defines time varying quantities that change during a simulation \tabularnewline 
    \textcolor{sthlmRed}{\textit{daePort\_t}} & Defines connection points between model instances for exchange of continuous quantities \tabularnewline 
    \textcolor{sthlmRed}{\textit{daeEventPort\_t}} & Defines connection points between model instances for exchange of discrete messages/events \tabularnewline
    \bottomrule
  \end{tabularx}
\end{table}
}
\end{frame}

\begin{frame}{Package \textsc{core} (cont'd)}
\scriptsize
{
\begin{table}
  \caption{The key modelling concepts in the \alert{core} package (cont'd).}
  \begin{tabularx}{\linewidth}{l>{\raggedright}X}
    \toprule
    \textcolor{sthlmRed}{\textbf{Concept}} & \textcolor{sthlmRed}{\textbf{Description}} \tabularnewline
    \midrule
    \textcolor{sthlmRed}{\textit{daePortConnection\_t}} & Defines connections between two ports \tabularnewline 
    \textcolor{sthlmRed}{\textit{daeEventPortConnection\_t}} & Defines connections between two event ports \tabularnewline
    \textcolor{sthlmRed}{\textit{daeEquation\_t}} & Defines model equations given in an implicit/acausal form \tabularnewline 
    \textcolor{sthlmRed}{\textit{daeSTN\_t}} & Defines state transition networks used to model discontinuous equations \tabularnewline
    \textcolor{sthlmRed}{\textit{daeOnConditionActions\_t}} & Defines actions to be performed when a specified condition is satisfied \tabularnewline
    \textcolor{sthlmRed}{\textit{daeOnEventActions\_t}} & Defines actions to be performed when an event is triggered on the specified event port \tabularnewline
    \textcolor{sthlmRed}{\textit{daeState\_t}} & Defines a state in a state transition network \tabularnewline 
    \textcolor{sthlmRed}{\textit{daeModel\_t}} & Represents a model \tabularnewline
    \bottomrule
  \end{tabularx}
\end{table}
}
\end{frame}

% \begin{frame}{Package core (cont'd)}
%     \begin{center}
%         \includegraphics[width=0.9\paperwidth]{Supplemental_Figure_S1.png}      
%     \end{center}
% \end{frame}

\begin{frame}{Package \textsc{core} - interface implementations}
    \begin{center}
        \begin{figure}
            \includegraphics[width=0.8\paperwidth]{Supplemental_Figure_S3.png}      
            %\caption{\alert{core} package interface implementations.}
        \end{figure}
    \end{center}
\end{frame}

\begin{frame}{Package \textsc{activity}}
\scriptsize
{
\begin{table}
  \caption{The key concepts in the \alert{activity} package.}
  \begin{tabularx}{\linewidth}{l>{\raggedright}X}
    \toprule
    \textcolor{sthlmRed}{\textbf{Concept}} & \textcolor{sthlmRed}{\textbf{Description}} \tabularnewline
    \midrule
    \textcolor{sthlmRed}{\textit{daeSimulation\_t}} & Defines ... \tabularnewline 
    \textcolor{sthlmRed}{\textit{daeOptimisation\_t}} &  \tabularnewline
    \bottomrule
  \end{tabularx}
\end{table}
}
\end{frame}

\begin{frame}{Package \textsc{solvers}}
\scriptsize
{
\begin{table}
  \caption{The key concepts in the \alert{solvers} package.}
  \begin{tabularx}{\linewidth}{l>{\raggedright}X}
    \toprule
    \textcolor{sthlmRed}{\textbf{Concept}} & \textcolor{sthlmRed}{\textbf{Description}} \tabularnewline
    \midrule
    \textcolor{sthlmRed}{\textit{daeDAESolver\_t}} & Defines ... \tabularnewline 
    \textcolor{sthlmRed}{\textit{daeLASolver\_t}} &  \tabularnewline
    \textcolor{sthlmRed}{\textit{daeNLPSolver\_t}} &  \tabularnewline
    \textcolor{sthlmRed}{\textit{daeIDALASolver\_t}} &  \tabularnewline
    \midrule
    \textcolor{sthlmRed}{\textit{daeMatrix\_t<typename FLOAT>}} &  \tabularnewline
    \bottomrule
  \end{tabularx}
\end{table}
}
\end{frame}

\begin{frame}{Package \textsc{solvers} - interface implementations}
    \begin{center}
        \begin{figure}
            \includegraphics[width=0.75\paperwidth]{Supplemental_Figure_S5.png}      
            %\caption{\alert{solvers} package interface implementations.}
        \end{figure}
    \end{center}
\end{frame}

\begin{frame}{Package \textsc{datareporting}}
\scriptsize
{
\begin{table}
  \caption{The key concepts in the \alert{datareporting} package.}
  \begin{tabularx}{\linewidth}{l>{\raggedright}X}
    \toprule
    \textcolor{sthlmRed}{\textbf{Concept}} & \textcolor{sthlmRed}{\textbf{Description}} \tabularnewline
    \midrule
    \textcolor{sthlmRed}{\textit{daeDataReporter\_t}} & Defines ... \tabularnewline 
    \textcolor{sthlmRed}{\textit{daeDataReceiver\_t}} &  \tabularnewline
    \bottomrule
  \end{tabularx}
\end{table}
}
\end{frame}

\begin{frame}{Package \textsc{datareporting} - interface implementations}
    \begin{center}
        \includegraphics[width=0.8\paperwidth]{Supplemental_Figure_S6.png}      
    \end{center}
\end{frame}

\begin{frame}{Package \textsc{log} and its interface implementations}
\scriptsize
{
\begin{table}
  \caption{The key concepts in the \alert{log} package.}
  \begin{tabularx}{\linewidth}{l>{\raggedright}X}
    \toprule
    \textcolor{sthlmRed}{\textbf{Concept}} & \textcolor{sthlmRed}{\textbf{Description}} \tabularnewline
    \midrule
    \textcolor{sthlmRed}{\textit{daeLog\_t}} & Defines ... \tabularnewline 
    \bottomrule
  \end{tabularx}
\end{table}
}
    \begin{center}
        \includegraphics[width=0.6\paperwidth]{Supplemental_Figure_S7.png}      
    \end{center}
\end{frame}

\begin{frame}{Package \textsc{units}}
\scriptsize
{
\begin{table}
  \caption{The key concepts in the \alert{units} package.}
  \begin{tabularx}{\linewidth}{l>{\raggedright}X}
    \toprule
    \textcolor{sthlmRed}{\textbf{Concept}} & \textcolor{sthlmRed}{\textbf{Description}} \tabularnewline
    \midrule
    \textcolor{sthlmRed}{\textit{unit}} & Defines ... \tabularnewline 
    \textcolor{sthlmRed}{\textit{quantity}} &  \tabularnewline
    \bottomrule
  \end{tabularx}
\end{table}
}
\end{frame}

\section{Developing models with DAE Tools} 

\begin{frame}{Overview}
\end{frame}

\section{Use Cases} 

\begin{frame}{Use Case 1 - High-Level Modelling Language}
\end{frame}

\begin{frame}{Use Case 2 - Low-Level DAE Solver}
\end{frame}

\begin{frame}{Use Case 3 - Embedded Simulator (back end)}
\end{frame}

\begin{frame}{Use Case 4 - Web Application / Web Service}
\end{frame}

% \appendix
% \begin{frame}[allowframebreaks]{References}
% \begin{thebibliography}{JModelica-2010}
% 
% \bibitem[Akesson et~al., 2010]{JModelica-2010}
% Akesson, J., Arzen, K.-E., Gafvert, M., Bergdahl, T., and Tummescheit, H.
%   (2010).
% \newblock Modeling and optimization with {Optimica} and {JModelica}.org -
%   languages and tools for solving large-scale dynamic optimization problems.
% \newblock {\em Comput. Chem. Eng.}, 34(11):1737--1749.
% 
% \bibitem[Andersson et~al., 2015]{Assimulo-2015}
% Andersson, C., Fuhrer, C., and Akesson, J. (2015).
% \newblock Assimulo: A unified framework for ode solvers.
% \newblock {\em Math. Comput. Simulat.}, 116(0):26 -- 43.
% 
% \bibitem[Balay et~al., 2015]{petsc}
% Balay, S., Abhyankar, S., Adams, M.~F., Brown, J., Brune, P., Buschelman, K.,
%   Dalcin, L., Eijkhout, V., Gropp, W.~D., Kaushik, D., Knepley, M.~G., McInnes,
%   L.~C., Rupp, K., Smith, B.~F., Zampini, S., and Zhang, H. (2015).
% \newblock {PETS}c users manual.
% \newblock Technical Report ANL-95/11 - Revision 3.6, Argonne National
%   Laboratory.
% 
% \bibitem[Barton and Pantelides, 1993]{Barton-and-Pantelides-1993}
% Barton, P.~I. and Pantelides, C.~C. (1993).
% \newblock {gPROMS - A Combined Discrete/Continuous Modelling Environment for
%   Chemical Processing Systems}.
% \newblock {\em Simul. Ser.}, 25:25--34.
% 
% \bibitem[Barton and Pantelides, 1994]{Barton-and-Pantelides-1994}
% Barton, P.~I. and Pantelides, C.~C. (1994).
% \newblock {Modeling of combined discrete/continuous processes}.
% \newblock {\em AIChE J.}, 40(6):966--979.
% 
% \bibitem[Bonami et~al., 2008]{Bonami-etal-2008}
% Bonami, P., Biegler, L.~T., Conn, A.~R., Cornu{\'e}jols, G., Grossmann, I.~E.,
%   Laird, C.~D., Lee, J., Lodi, A., Margot, F., Sawaya, N., and W{\"a}chter, A.
%   (2008).
% \newblock {An algorithmic framework for convex mixed integer nonlinear
%   programs}.
% \newblock {\em Discrete Optim.}, 5(2):186--204.
% \newblock In Memory of George B. Dantzig.
% 
% \bibitem[Brook et~al., 1988]{Brook-etal-1988}
% Brook, A., Kendrick, D., and Meeraus, A. (1988).
% \newblock {GAMS, a User's Guide}.
% \newblock {\em SIGNUM Newsl.}, 23(3-4):10--11.
% 
% \bibitem[Eaton et~al., 2015]{octave}
% Eaton, J., Bateman, D., Hauberg, S., and Wehbring, R. (2015).
% \newblock {\em {GNU Octave} version 4.0.0 manual: a high-level interactive
%   language for numerical computations}.
% 
% \bibitem[Elmqvist, 1978]{Elmqvist-1978}
% Elmqvist, H. (1978).
% \newblock {\em {A Structured Model Language for Large Continuous Systems}}.
% \newblock PhD thesis, Department of Automatic Control, Lund University, Sweden.
% 
% \bibitem[Fritzson et~al., 2005]{OpenModelica-2005}
% Fritzson, P., Aronsson, P., Lundvall, H., Nystrom, K., Pop, A., Saldamli, L.,
%   and Broman, D. (2005).
% \newblock The openmodelica modeling, simulation, and development environment.
% \newblock SIMS - Scandinavian Simulation Society.
% 
% \bibitem[Fritzson and Engelson, 1998]{Fritzson-and-Engelson-1998}
% Fritzson, P. and Engelson, V. (1998).
% \newblock {Modelica --- A unified object-oriented language for system modeling
%   and simulation}.
% \newblock In Jul, E., editor, {\em {ECOOP{\rq}98 --- Object-Oriented
%   Programming}}, volume 1445 of {\em {Lecture Notes in Computer Science}},
%   pages 67--90. Springer Berlin Heidelberg.
% 
% \bibitem[Hedengren et~al., 2014]{APMonitor-2014}
% Hedengren, J.~D., Shishavan, R.~A., Powell, K.~M., and Edgar, T.~F. (2014).
% \newblock Nonlinear modeling, estimation and predictive control in apmonitor.
% \newblock {\em Comput. Chem. Eng.}, 70:133 -- 148.
% \newblock Manfred Morari Special Issue.
% 
% \bibitem[Hindmarsh et~al., 2005]{Hindmarsh-etal-2005}
% Hindmarsh, A.~C., Brown, P.~N., Grant, K.~E., Lee, S.~L., Serban, R., Shumaker,
%   D.~E., and Woodward, C.~S. (2005).
% \newblock {SUNDIALS: Suite of Nonlinear and Differential/Algebraic Equation
%   Solvers}.
% \newblock {\em ACM Trans. Math. Softw.}, 31(3):363--396.
% 
% \bibitem[Johnson, 2015]{nlopt}
% Johnson, S.~G. (2015).
% \newblock {The NLopt nonlinear-optimization package}.
% 
% \bibitem[Li, 2005]{Li-2005}
% Li, X.~S. (2005).
% \newblock {An Overview of SuperLU: Algorithms, Implementation, and User
%   Interface}.
% \newblock {\em ACM Trans. Math. Softw.}, 31(3):302--325.
% 
% \bibitem[Li et~al., 2014]{Li-etal-2014}
% Li, Y., El~Gabaly, F., Ferguson, T.~R., Smith, R.~B., Bartelt, N.~C., Sugar,
%   J.~D., Fenton, K.~R., Cogswell, D.~A., Kilcoyne, D. A.~L., Tyliszczak, T.,
%   Bazant, M.~Z., and Chueh, W.~C. (2014).
% \newblock {Current-induced transition from particle-by-particle to concurrent
%   intercalation in phase-separating battery electrodes}.
% \newblock {\em Nat. Mater.}, 13(12):1149--1156.
% 
% \bibitem[Morton, 2003]{Morton-2003}
% Morton, W. (2003).
% \newblock {Equation-Oriented Simulation and Optimization}.
% \newblock pages 317--357. Indian National Sciences Academy.
% 
% \bibitem[Piela et~al., 1991]{Piela-etal-1991}
% Piela, P.~C., Epperly, T.~G., Westerberg, K.~M., and Westerberg, A.~W. (1991).
% \newblock {ASCEND: an object-oriented computer environment for modeling and
%   analysis: The modeling language}.
% \newblock {\em Comput. Chem. Eng.}, 15(1):53--72.
% 
% \bibitem[Sala et~al., 2006]{amesos-2006}
% Sala, M., Stanley, K., and Heroux, M. (2006).
% \newblock {Amesos: A Set of General Interfaces to Sparse Direct Solver
%   Libraries}.
% \newblock In {\em {Proceedings of PARA'06 Conference, Umea, Sweden}}.
% 
% \bibitem[Schenk et~al., 2007]{Schenk-etal-2007}
% Schenk, O., W{\"a}chter, A., and Hagemann, M. (2007).
% \newblock {Matching-based preprocessing algorithms to the solution of
%   saddle-point problems in large-scale nonconvex interior-point optimization}.
% \newblock {\em Comput. Optim. Appl.}, 36(2-3):321--341.
% 
% \bibitem[{Scilab Enterprises}, 2015]{scilab}
% {Scilab Enterprises} (2015).
% \newblock {Scilab: Free and Open Source software}.
% 
% \bibitem[{The MathWorks, Inc.}, 2015]{matlab}
% {The MathWorks, Inc.} (2015).
% \newblock {MATLAB}.
% 
% \bibitem[W{\"a}chter and Biegler, 2006]{Wachter-and-Biegler-2006}
% W{\"a}chter, A. and Biegler, L.~T. (2006).
% \newblock {On the implementation of an interior-point filter line-search
%   algorithm for large-scale nonlinear programming}.
% \newblock {\em Math. Program.}, 106:25--57.
% 
% \bibitem[Walther and Griewank, 2012]{Walther-and-Griewank-2012}
% Walther, A. and Griewank, A. (2012).
% \newblock {\em {Getting started with ADOL-C}}.
% \newblock Chapman-Hall CRC Computational Science.
% 
% \bibitem[{Waterloo Maple, Inc.}, 2015]{maple}
% {Waterloo Maple, Inc.} (2015).
% \newblock {Maple}.
% 
% \bibitem[{Wolfram Research, Inc.}, 2015]{mathematica}
% {Wolfram Research, Inc.} (2015).
% \newblock {Mathematica}.
% 
% \end{thebibliography}
% 
% \end{frame}

\end{document}